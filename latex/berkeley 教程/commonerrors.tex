\chapter{Common Errors}
\label{cha:common-errors}

If you have every compiled a \LaTeX{} document, chances are high you
have received a few error messages.  Sometimes they come from
something as stupid and as easy to fix as forgetting a parenthesis or
forgetting to end an environment.  There are also a lot more cases
where you have no idea what you have done wrong and it takes you a
long time to find or even understand your error.

The purpose of this is to explain some of the common errors that may
happen when compiling a \LaTeX{} document and suggestions for what is
probably going on and how to debug your document.

\section{The Form of an Error}
\label{sec:form-an-error}

There are two forms of errors: \LaTeX{} errors and \TeX{} errors.  In
both types of errors, the part after the error message will tell you
where the error occurred.  An example:
\begin{verbatim}
l.15 <offending text>
\end{verbatim}
The \texttt{l.15} tells you what line the error occurred on and the
text will tell you the text that caused the error.

\subsection{\LaTeX{} Errors}
\label{sec:latex-errors}

The general form of an error in \LaTeX{} is shown below:
\begin{verbatim}
! LaTeX error: <error message>

See the LaTeX manual or LaTeX Companion for
explanation.
Type H <return>   for immediate help.
...
\end{verbatim}

The \texttt{!} lets you know that the error has occurred.  The error
message will tell you what type of error you have committed.  After
the ellipses, you will find the line at which the error occurred and
the text that caused the error (or at least the text where \LaTeX{}
found the error).

\subsection{\TeX{} Errors}
\label{sec:tex-errors}

Errors may also have the following form:
\begin{verbatim}
! <error message>
\end{verbatim}

These errors are formatted differently because they are error messages
that came from \TeX{} instead of \LaTeX{}.  After the error, you will
still find the line that the error occurred in and the text of the
error.

\section{Warnings}
\label{sec:warnings}

There are some error messages that are just warnings and will not stop
or change the compilation of the document.  Chances are you have seen
them many times.

\subsection{Underfull}
\label{sec:underfull}

The following error results when a line does not extend the width of
the page, something \LaTeX{} always tries to accomplish:
\begin{verbatim}
Underfull \hbox (badness 10000) in paragraph at lines
104--107
\end{verbatim}

This error message is just a warning and is not something to worry
about.  For the most part, when a line does not span the width of the
page, it is because you have written something that you want to only
cover part of the page.

\subsection{Overfull}
\label{sec:overfull}

The following error results when a line extends beyond the width of
the page:
\begin{verbatim}
Overfull \hbox (16.04988pt too wide) in paragraph at
lines 30--31 [] [] \OT1/cmtt/m/n/12 I'm trying to put
way too much text into a line in my document.
\end{verbatim}
Usually this error comes form when you are using the \texttt{verbatim}
package because it will not move to the next line if your text does
not go to the next line.  The easiest way to fix this is to find the
place in your document where this is occurring and change the text so
that it fits to the page.

This error will still show up if the text is still on the page but
outside of the width of text that \LaTeX{} has set.  In this case, you
are welcome to fix things so that the error does not show up or you
can leave the text as it is.

\subsection{References}
\label{sec:references-1}

The following warnings occur when references are changed when \LaTeX{}
was compiled:
\begin{verbatim}
LaTeX Warning: Label(s) may have changed. Rerun to get
cross-references right.

LaTeX Warning: There were undefined references.

LaTeX Warning: Reference `name' on page 1 undefined on
input line 15.
\end{verbatim}

The way to fix these errors is to recompile the document again to
correct the page numbers.  Sometimes it is necessary to recompile the
document twice to fix this error.  You also may have defined a
reference wrong, so you should check to make sure your label is
correct.

\section{Beginning and Ending}
\label{sec:beginning-ending}

\subsection{Begin Ended by End}
\label{sec:begin-ended-end}

This type of error occurs when each environment is not correctly
started and ended.  When you are missing an \verb|\end| command, the
following error will show up:
\begin{verbatim}
! LaTeX Error: \begin{enumerate} on input line 23
ended by \end{document}.
\end{verbatim}

To fix this, you need to end the environment mentioned in the error
with the appropriate command.

When you are missing a \verb|\begin| command, the following will appear:
\begin{verbatim}
! LaTeX Error: \begin{document} ended by
\end{itemize}.
\end{verbatim}

To fix this, you basically do the same thing as before, correctly
beginning the environment mentioned in the error with the appropriate
command.

\subsection{End Occurred Inside a Group}
\label{sec:end-occurred-inside}

The following error message will show up at the end of compiling a
file if an environment is begun that is not ended:
\begin{verbatim}
(\end occurred inside a group at level <n>)
\end{verbatim}

To fix this error, make sure you end the environment that was begun.
The previous error is more helpful in finding the \verb|\begin|
statement.

\subsection{Ended by End of Line}
\label{sec:ended-end-line}

The following error will occur when you try to place a command inside
a section heading:
\begin{verbatim}
! LaTeX Error: \verb ended by end of line.

See the LaTeX manual or LaTeX Companion for
explanation.
Type H <return>   for immediate help.
...
\end{verbatim}

There will be many errors of the same type for this mistake.  In order
to find where you put the command, look in the output file and find
the last heading that shows up.

\subsection{Missing Begin Document}
\label{sec:miss-begin-docum}

This error is self-explanatory:
\begin{verbatim}
! LaTeX Error: Missing \begin{document}
\end{verbatim}

\section{Errors Usually Caused by Bad Spelling}
\label{sec:errors-usually-caus}

\subsection{Unknown Control Sequence}
\label{sec:unkn-contr-sequ}

This error results when you use a command (something that starts with a
\texttt{$\backslash$}) that is not recognized by \LaTeX{}:
\begin{verbatim}
! Undefined control sequence.
\end{verbatim}

Usually this error results from spelling a command incorrectly.  Go to
the line that is indicated by the error and fix the command.

\subsection{Environment Undefined}
\label{sec:envir-undef}

This error results when you begin an environment with a \verb|\begin|
command that is not recognized:
\begin{verbatim}
! LaTeX Error: Environment verbatin undefined.
\end{verbatim}

Usually you have just spelled your environment incorrectly, so you
just need to fix it.

\subsection{Bad File Name}
\label{sec:bad-file-name}

This error results when you have mistyped the command \texttt{latex}
or do not have \LaTeX{} installed on your computer:
\begin{verbatim}
Bad command or file name
\end{verbatim}
To fix this, correctly spell the command to compile your file or make
sure that \LaTeX{} is correctly installed on your computer.

\subsection{Cannot Find File Name}
\label{sec:cannot-find-file}

This error occurs when you try to compile a file that the computer
cannot find:
\begin{verbatim}
! I can't find file `sample'.
<*> sample

Please type another input file name:
\end{verbatim}

To fix this error, make sure you have spelled the file name
correctly.  You also may be in the wrong directory to compile the
file, so check to make sure you are in the same directory as your
file.

\section{Fatal Errors}
\label{sec:fatal-errors}

\subsection{Runaway Argument}
\label{sec:runaway-argument}

This error happens when a paragraph ends before a command's argument
is done (i.e., \LaTeX{} thinks that there is a missing \}):
\begin{verbatim}
Runaway argument?
\end{verbatim}

To fix this, you should use a different command to accomplish what you
are trying to do.  An example of this is to use \verb|\bfseries|
instead of \verb|\bftext| to make bold text in more than one
paragraph.

This error can also be caused by a missing mandatory argument to a
command.

\subsection{Just an *}
\label{sec:just-an-asterisk}

This error normally occurs when you do not end your document with \\ \verb|\end{document}|:
\begin{verbatim}
*
\end{verbatim}

If you are prompted to enter something in, it is best to enter
\begin{verbatim}
\end{document}
\end{verbatim}
and hope it works.  Be sure to end your document
with the appropriate command.

\subsection{Emergency Stop}
\label{sec:emergency-stop}

This error happens when \LaTeX{} will stop trying to compile your
document due to a serious error:
\begin{verbatim}
! Emergency stop.
\end{verbatim}
To fix this error, you will need to figure out what caused it to stop
compiling.  Chance are you forgot to end your document with
\verb|\end{document}|, but there might also be another reason for the
emergency stop.

\subsection{Please Type a Command or Say End}
\label{sec:please-type-command}

This error happens when your file has ended prematurely:
\begin{verbatim}
(Please type a command or say `\end')
\end{verbatim}

The best way to deal with this type of error is to type
\begin{verbatim}
\end
\end{verbatim}
or
\begin{verbatim}
\end{document}
\end{verbatim}
in the case that the absence of that command
caused the error.  Usually if you have ended your document correctly,
the error will result from a missing \} or forgetting to end a
verbatim environment.

\section{Graphics Errors}
\label{sec:graphics-errors}

\subsection{Too Many Unprocessed Floats}
\label{sec:too-many-unprocessed}

This error occurs when figures or tables (i.e., floats) have not been
typeset:
\begin{verbatim}
! LaTeX Error: Too many unprocessed floats.
\end{verbatim}

\LaTeX{} can only have so many floats waiting to be typeset.  In order
to fix this error, make sure that you are placing your floats where
you want them (with a \texttt{[h]} option) and not wanting too many on
one page in sequence.  Using the command \verb|\clearpage| can be very
useful in distributing floats correctly.

\subsection{Unknown Graphics Extension}
\label{sec:unkn-graph-extens}

The following error occurs when you try to use a type of graphic that
is not supported by the type of file that you are producing:
\begin{verbatim}
! LaTeX Error: Unknown graphics extension: .gif
\end{verbatim}

In order to fix this error, you should change your graphics to the
types that are supported by the type of file you are outputting or you
will need to include the correct package to deal with that type of
graphic.  Sometimes you may have named the graphic poorly so that
\LaTeX{} will not recognize it as a graphic file.

\subsection{Division by Zero}
\label{sec:division-zero}

The following error occurs when the height of a graphic object is zero:
\begin{verbatim}
! Package graphics Error: Division by 0.
\end{verbatim}

This is usually caused when you rotate an object with zero depth so
that its height becomes zero.  The best way to fix this is to use the
keyword \texttt{totalheight} instead of \texttt{height}.

\section{Math Errors}
\label{sec:math-errors}

\subsection{Display Math Should End With \$\$}
\label{sec:display-math-should}

This error occurs when the displaymath or equation mode is ended incorrectly:
\begin{verbatim}
! Display math should end with $$
\end{verbatim}

To fix this error, make sure that you end the displaymath or equation
mode correctly (ending them with a \$ is not acceptable).

\subsection{Bad Math Environment Delimiter}
\label{sec:bad-math-environment}

This error occurs when you do not have your delimiters correct in math
mode:
\begin{verbatim}
! LaTeX Error: Bad math environment delimiter.
\end{verbatim}

Usually this occurs when you forget to match a right delimiter with
every left delimiter.  This error may also happen when you forget to
end an array.

\subsection{Missing Right}
\label{sec:missing-right}

This error occurs when you have a missing right parenthesis:
\begin{verbatim}
! Extra \right.
\end{verbatim}

To fix this, you either need to add a \verb|\right| command or you
need to end an array.

\subsection{Missing Delimiter}
\label{sec:missing-delimiter}

This error message occurs when a delimiter is missing:
\begin{verbatim}
! Missing delimiter (. inserted).
\end{verbatim}

To fix this error, you need to make sure that you have a right
delimiter for every left delimiter.  If you do not want a right
delimiter matching a left delimiter, you need to use ``.''\ to not
have an error message show up.

\subsection{Missing \$ Inserted}
\label{sec:missing--inserted}

The following error occurs when you try to use a character that can
only be used in math mode, like \_ or \^{}:
\begin{verbatim}
! Missing $ inserted
\end{verbatim}

To fix this error, make sure you change the character to what it
should be in text mode.

\section{Tabular Environment Errors}
\label{sec:tabul-envir-errors}

\subsection{Misplaced Alignment Tab Character \&}
\label{sec:mispl-alignm-tab}

This error occurs when you use \texttt{\&} and when you are not in a
tabular environment:
\begin{verbatim}
Misplaced alignment tab character &
\end{verbatim}

To fix this error, you need to use \verb|\&| to make a \&.

\subsection{Extra Alignment Tab}
\label{sec:extra-alignment-tab}

This error occurs when you use too many tabs for the number of columns
in a table:
\begin{verbatim}
! Extra alignment tab has been changed to \cr
\end{verbatim}

The result of this error is that a new row is formed where the extra
tab was.  You should go back and fix your table so that the correct
number of items in each row would show up.

\subsection{Argument Has an Extra \}}
\label{sec:argument-has-an}

These errors happen when an incorrect number of arguments to a tabular
environment have been specified:
\begin{verbatim}
! Argument of \cline has an extra }.

! Argument of \multicolumn has an extra }.
\end{verbatim}

To fix this error, make sure your arguments to the tabular environment
are correct.

\section{Errors With Lists}
\label{sec:errors-with-lists}

\subsection{Missing Item}
\label{sec:missing-item}

This error occurs when there is plain text in an environment that
takes items:
\begin{verbatim}
! LaTeX Error: Something's wrong--perhaps a missing
\item.
\end{verbatim}

To fix this error, make sure the plain text is changed into an item.

\subsection{Too Deeply Nested}
\label{sec:too-deeply-nested}

This error occurs when there are too many lists for \LaTeX{} to
handle:
\begin{verbatim}
! LaTeX Error: Too deeply nested
\end{verbatim}

\LaTeX{} can only handle four levels of one type of list and six
levels of different types of lists.  To fix this, you need to use less
levels of lists or define your own list environment.

\section{Miscellaneous Errors}
\label{sec:miscellaneous-errors}

\subsection{Only Used in the Preamble}
\label{sec:only-used-preamble}

This error occurs when you place a command in the body of a \LaTeX{}
document that should be placed in the preamble:
\begin{verbatim}
! LaTeX Error: Can be used only in the preamble.
\end{verbatim}

To fix this error, just move the command to the preamble.

\subsection{There Is No Line/Page Here to End}
\label{sec:there-no-linepage}

This error occurs when you incorrectly use the commands that make a
new line or a new page:
\begin{verbatim}
! LaTeX Error: There's a no line here to end.
\end{verbatim}

You may just leave the command that is making a new line in place or
you can take it out.  Here, \LaTeX{} is just trying to make sure that
everything looks nice.

\subsection{Command Already Defined}
\label{sec:comm-alre-defin}

This error occurs when you try to define a command that already exists:
\begin{verbatim}
! LaTeX Error: Command ... already defined.
\end{verbatim}

To fix this, you need to define your command differently.

\subsection{Missing Number}
\label{sec:missing-number}

This error is made when a number is expected as an argument and one is
not provided:
\begin{verbatim}
! Missing number, treated as zero.
\end{verbatim}

To fix this error, you need to find where a number is expected so that
you can provide the correct one.

%%% Local Variables: 
%%% mode: latex
%%% TeX-master: "reader"
%%% End: 