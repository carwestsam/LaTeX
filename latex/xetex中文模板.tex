
%文件的类型

\documentclass[12pt,a4paper]{ctexrep} 
\usepackage{titlesec}

%设置页边距
\usepackage[top=1in,bottom=1in,left=1.25in,right=1in]{geometry}

%设置行距
%在使用ctex时注释掉
%\renewcommand{\baselinestretch}{1.62}


%设置标题格式
\titleformat{\section}{\Large\AdobeHei}{\thesection}{1em}{}

\usepackage{fontspec}
\XeTeXlinebreaklocale "zh"
\XeTeXlinebreakskip = 0pt plus 1pt



\newfontfamily\AdobeHei{Adobe Heiti Std}
\newfontfamily\AdobeSong{Adobe Song Std}
\newfontfamily\AdobeKai{Adobe Kaiti Std}


\setmainfont[BoldFont=Adobe Heiti Std]{Adobe Song Std}
\setsansfont[BoldFont=Adobe Heiti Std]{AdobeKaitiStd-Regular}
\setmonofont{Times New Roman}


%重定义摘要的名称
%\renewcommand{\abstractname}{摘要}
%\renewcommand{\chaptername}{章}
%\renewcommand{\contentsname}{mulu}

% 设置段落内缩
\parindent=2em
\makeatletter
\let\@afterindentfalse\@afterindenttrue
\@afterindenttrue
\makeatother
\setlength{\parindent}{2em}%中文缩进两个汉字位

% ToC:
\usepackage{titletoc}

%用来处理ctex中目录显示中的重叠问题
%实际上是调整了目录的相关设置
%\dottedcontents{chapter}[3.4em]{\bf}{3em}{5pt}
\usepackage{tocloft}% http://ctan.org/pkg/tocloft
\setlength{\cftchapnumwidth}{4em}
\setlength{\cftsubsecnumwidth}{4em}% Set length of number width in ToC for \subsection

%关于在Adobe中侧边超链接目录的设置
\newcommand\frontmatterchaptoc{%
\titlecontents{chapter}
  [1.5em]{\addvspace{\baselineskip}}
  {\contentslabel{1.5em}\MakeUppercase}
  {\hspace*{-1.5em}}% \MakeUppercase removed
  {\titlerule*[1pc]{.}\contentspage}
}

\usepackage[bookmarks=true,colorlinks,linkcolor=black]{hyperref}
\hypersetup{
  unicode,  
   %linktocpage,
}
%\usepackage{bookmark}

\begin{document}

\title{数据结构第五次上机作业}

\date{\today}

\maketitle
\tableofcontents


\begin{abstract}

一个精简版的c语言语法分析器。可以完成90\%的C语言的语法分析。

\end{abstract}
\chapter{总体介绍}
\section{主要功能}
从input.txt中读入需要解释的代码。解释器将会生成parser\_st.dot.pdf这个文件。文件中包含了代码所构成的语法树。\par
同时生成的文件debug.txt中包含了代码的调试信息。由于时间紧迫,没有能将出错信息精简,并且输出到标准输出流中。
目前的调试信息依然十分冗长,需要经过处理才能阅读。\\
 可以识别大多数的C语言语法。不能读入字符串,不能使用typedef语句,没有宏命令。数字只有一般的十进制表示法,不能加前缀或者后缀。\\
\section{编程环境和工具}
下面列出了在编写代码时使用的相关环境。
\paragraph {操作系统:}
CentOS 6.4 final
\paragraph{编译器:}
gcc version 4.4.7 20120313 (Red Hat 4.4.7-3) (GCC)
\paragraph{文本编辑器:}
VIM - Vi IMproved 7.2 (2008 Aug 9, compiled Apr  5 2012 10:17:30)
\paragraph{pdf生成工具}
dot - graphviz version 2.26.0 (20091210.2329)
\paragraph{pdf查看工具}
Google Chrome 26.0.1410.63
\paragraph{版本管理工具}
git version 1.7.1
\section{简单的示例}
对于一个文件:

\par 将生成下面的图片:
\begin{figure}[h]

\caption{样例输出}
\label{样例输出1}
\end{figure}
\par 以及下面的括号表达式:

\chapter{编写过程}
\section{早期版本}
我第一次写了一个简单的词法读入代码。可以完成基本的符号,关键字等读入。C语言实现。但是由于C语言本身的能力有限,所以不能很好的完成指定的任务。
\section{中期版本}
将早期版本的代码使用c++语法封装,作为词法分析部分。
采用了部分LR(1)和LL(1)相结合的方法来读入语法。
\subsection{编写风格的变化}
\subsubsection{编译开关}
	大量使用编译开关来帮助调试,控制输出信息。

\subsubsection{多个文件}
	由于代码量的急剧上升,将大段的代码放到了多个文件中,对于每一个文件有相应的测试代码。方便后期追踪调试。
\subsubsection{c++语言实现}
	早期版本是使用C语言实现,这个版本换成c++,大量使用了模板等高级特性。
\subsection{LR语法}
考虑到LL(1)代码的局限性,将一部分语法读入使用LR来实现。
在纸上写一个LR分析过程是非常简单的。但是要变成代码,还有很长的路要走。
下面展示两个代码片段。第一个是归约的部分代码,第二个是LR的进行代码。


\subsection{LL语法}
递归下降来处理一般的代码。比如循环语句和一般的括号匹配。下面是一段示例代码。

\subsection{缺点和不足}
\paragraph{接口过多}
我在每一层都添加了过多的信息,导致整个代码有三分之一的代码在处理层与层之间的关系。这浪费了我自己大量的工作编写。
而且调试起来非常冗杂。我不应该学习编译原理书后面的java代码,因为java和c++的区别确实很大,实现方法也会有很大的区别。
\paragraph{调试信息编写不合理}
初期没有意识到会有大量的调试代码,所以调试信息输出得很随意。但是到了编写的后期,发现需要大量的调试信息,但是修改起来非常不方便。而且所有的调试信息没能得到很好的管理。
\chapter{代码介绍}
\section{总体架构}
分析分成了三个大块。预处理,词法分析,和语法分析。
\begin{table}[h]
\caption{代码文件表}

\centering
\begin{tabular}[h]{|c|c|}
\hline
文件名 & 功能 \\
core.cpp	&  主要的类定义 \\
number.cpp	& 处理数值的定义\\
error.cpp	& 异常信息的处理 \\
pre\_read.cpp & 预处理的定义 \\
pre\_deal.cpp & 预处理 \\
grammar.cpp & 语法和词法的定义 \\
token.cpp	& 词法的定义 \\
parser.cpp & 定义语法树的结构 \\
parser\_build.cpp & 进行语法分析 \\
\hline

\end{tabular}
\end{table}
\section{处理流程}
\subsection{预处理}
由预处理来读入,将整个文本读入内存之中。去除不能识别的字符,比如“¥”等。将不能识别的字符变成空格进行读入。同时将文中的注释清除掉。记录每一个字符的行号和位置。将整个文本变成一个个字符的包,传递给词法分析器。
\subsection{词法分析}
从预处理中读入处理过的字符信息。识别出其中的关键字和符号。将数字转换成数值类型。忽略空格和换行。
\subsection{语法分析}
按照附录二中的语法定义去实现EBNF语法。这个语法定义是陈伦荣同学提供的,我们小组分了两条线,陈伦荣使用antlr生成代码,我则是自己动手实现。\par
将关键字和符号比成语法规则所定义的语法树。由于语法定义中包含了大量的无用的非终结符号,以及无法递归实现语义的语法,所以我没有生成AST树。而是更加单纯的去实现一个语法树。将AST树的生成放到语义分析的部分去生成。\par
\section{元语言设计}
在编写过程中,我大量使用了由宏实现的元语言设计。极端地减少了代码长度,大大增加了代码的可读性和和维护性。
对于下面的一个生成语法树的函数,如果是普通实现的话,描述如下:

但是使用了元语言设计方法之后,代码如下:

可以发现,上面的代码甚至连一点程序实现都没有,看起来非常清晰。这让代码的逻辑和控制部分分离,减少了出错的可能型,也方便了自己的调试及控制。
\section{运行方法}

输入的代码放入input.txt中。
编译运行parser\_build.cpp即可。
编译生成的树会以dot文件的形式输出。这个文件需要使用graphviz工具编译生成图片或者pdf文档。

\chapter{相关说明}
\section{输入输出信息}
对于一个输入文件,代码会有以下几个输出。

\subsection{预处理信息}

\subsection{词法信息}

\subsection{语法树信息}

\section{存在的问题}
在整体的编写过程中,没能做到将每一个部分分割开来。但是考虑到如果多次输出输入的话,每次读入都需要解释上一次的输出信息。相当与写了一个语法分析器。这其实是一个鸡生蛋蛋生鸡的问题,所以最终没有将输入输出分割开来。但是这导致的代码的独立性不够好。\par
在前两个版本的代码中,在每次的中间过程使用了大量的接口,导致的代码效率及其低下和实现工作量巨大。第三个版本的代码虽然和第二部版本的代码的总量相近,但是实现语法功能确强大了十多倍。第三个版本的代码依然有许多地方可以优化。比如大量使用的字符串的匹配来减少判断语句的工作量。这实际上会给整个代码运行时间乘上10左右的常数。如果有时间的话,可以考虑使用自动机来大大提高这个部分的效率。\par
语法功能过于强大,实现了90\%左右的C语言文法。导致后期的语义工作量更加巨大。没有办法在规定时间内完成任务。\par
\section{总结}
这次的解释器大作业,我总共写了大约5000行的有效代码。其中最终版本大约两千行。总的来说,代码量很大,思维的难度也很高。有许许多多的地方可以不断地挑战自己。不断地尝试更好的实现方法,更加高效的编写方式。很多时候书上的代码并不能够给出很好的解决自己的问题。将书上的内容实现,和将书本中的内容弄懂完全是两回事。实现中会遇到大量的语言及操作系统相关的内容。对于大量的维护和调试自己也是第一次。我使用了编译开关来辅助调试。使用了git来进行版本管理。对于我以后编写大型程序打下了基础。\par
工作量巨大可能和我选用的编程语言相关。最早的版本是c语言。由于c语言实在是过于简陋,实现非常痛苦。到后来改用C++语言,有效地减少了自己的工作量(至少少了两倍)。但是如果我一开始的时候就能使用secheme等高级语言的话,或许我能够节省大部分的时间。工欲善其事,必先利其器。希望自己以后可以做得更好。\par


\chapter{附录一 其他代码}
\section{早期代码}

\chapter{附录二 最终代码}

\section{语法定义}
这是一个使用EBNF语法描述的简化版C语言文法。没有typedef语法。只有基本的十进制数字。不支持字符输入。


\section{头文件}

core申明了所有的类,定义一些常量。定义了文件打开和关闭的函数。\par
symbols定义了我将会使用的一些字符常量。不过由于代码并没有完工,许多内容只是定义了,实际上没有使用。
注意symbols[]和keywords[]两个数组中的顺序非常重要。我是顺序检测各个关键字,如果检测到了就会进行读入。
举个例子,要读入一个符号"<<",如果symbols中,"<"在"<<"之前,我会读入两个"<",从而无法读取到位移操作符。




\section{输出}
两个文件error和op,表示了出错信息和输出信息。


\section{数字}
处理数字的计算,数字类型的转换等。

\section{预处理}
pre\_read定义了一个CharString类,用来处理输入的程序。将输入文件变成一个字符流。\par
pre\_deal对输入文件进行预处理。去除不能识别的字符,去除注释内容。


\section{语法}
\subsection{数据结构定义}
Grammar定义了语法树的结构。用来表示语法中的一个节点。是Token类和Parser类的父类。

\subsection{词法}
Token将读入的字符流变成了token,读入了数字,标识符和关键字。

\subsection{语法}
Parser定义了语法树的结构。parser\_build.cpp是语法树的构建代码。
\end{document}